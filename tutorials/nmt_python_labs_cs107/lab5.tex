\documentclass[11pt]{cselabheader}

\newcommand{\thelabnumber}{5}
\newcommand{\thetitle}{Strings and Tuples}
\newcommand{\theauthor}{CSE/IT 107L}

% Write title and author
\title{\thetitle}
\author{\theauthor}
\date{NMT Department of Computer Science and Engineering}

\fancyhead[R]{Lab \thelabnumber: \thetitle}
\fancyhead[L]{\theauthor}

\makehyperref
\makeindex[title=Index of New Functions and Methods, intoc]

\begin{document}

\pagenumbering{roman}
\maketitle
\hrule

\begin{multicols}{2}
``All thought is a kind of computation.''
\begin{flushright}
--- D. Hobbes
\end{flushright}

``It [programming] is the only job I can think of where I get to be both an
engineer and artist. There's an incredible, rigorous, technical element to it,
which I like because you have to do very precise thinking. On the other hand, it
has a wildly creative side where the boundaries of imagination are the only real
limitation.''
\begin{flushright}
--- A. Hertzfeld
\end{flushright}

\section*{Introduction}
This lab will introduce two new Python values that are very similar to
lists: tuples and strings. You have seen strings before, but now you
will learn to manipulate them. Strings and tuples support indexing,
slicing, membership testing with \pythoninline{in}, equality checks,
and other similar features. The main difference is that they are
immutable, meaning that the elements at each index cannot be changed.

The lab also introduces keyword arguments, which are useful for multi-purpose
functions or for those with arguments that have default values. With a
more detailed knowledge of strings, you will be able to write documentation
for functions in the form of docstrings, which will be visible using the
\pythoninline{help()} function.

\begin{figure}[H]
  \centering
  \includegraphics[width=\linewidth]{img/xkcd_1024.png}
  \caption{\url{http://xkcd.com/1024}}
\end{figure}
\end{multicols}

\tableofcontents

\newpage
\pagenumbering{arabic}

\section{Tuples: Fixed-Length, Immutable Ordered Collections}

Tuples\index{tuples} work a lot like lists, except they cannot be
modified; they are \emph{immutable}\index{immutable}. Instead of brackets
\pythoninline{[]}, tuples are delimited by writing parentheses
\pythoninline{()} around the elements. Tuples support
slicing\index{slicing tuples}, just as lists do.

\begin{pyconcode}
>>> food = ('eggs', 'bananas', 'lemonheads')
>>> food[1]
'bananas'
>>> food[1:3]
('bananas', 'lemonheads')
>>> food[1] = 'steak' # tuples are immutable!
Traceback (most recent call last):
  File "<stdin>", line 1, in <module>
TypeError: 'tuple' object does not support item assignment

\end{pyconcode}

Notice Python raises a \pythonindex{TypeError} with message
```tuple' object does not support item assignment'' if the code
attempts to assign to an item within a tuple.
If a tuple contains a mutable value, such as a list, then that
element can be modified without restriction.

We can nest tuples and lists interchangeably:

\begin{pyconcode}
>>> points = [(1, 5), (3, 5), (0, 2), (1, 1),  # shift+enter for multi-line
...           (2, 1), (3, 1), (4, 2)]
>>> points[0]
(1, 5)
>>> fubar = ([1, 2, 3], [4, 5, (7, 8, 9)], (10, 11))
>>> fubar
([1, 2, 3], [4, 5, (7, 8, 9)], (10, 11))

\end{pyconcode}

Tuples, like lists, can also be empty. Use \pythonindex{len()} to find the
length of a tuple:

\begin{pyconcode}
>>> foo = ()
>>> len(foo)
0

\end{pyconcode}

We can also define a tuple by simply separating values by commas:
Conversely, we can \emph{unpack} tuples\index{unpack tuples}, which
means assigning each element of a tuple to a variable. For example:

\begin{pyconcode}
>>> food = 'trail mix', 'nothing'
>>> tylerfood, chrisfood = food
>>> # equivalent to, but shorter than: tylerfood = food[0]; chrisfood = food[1]
>>> tylerfood
'trail mix'
>>> chrisfood
'nothing'

\end{pyconcode}

Unpacking is useful for assigning a fixed number of return values from a function.

\section{Defining, Using, and Transforming Strings}

You have seen strings\index{strings} in Python before. They are
sequences of characters enclosed by either double quotes or single
quotes; for example:

\begin{pyconcode}
>>> s = "I'm a string."
>>> print(s)
I'm a string.
>>> r = 'I am also a string.'
>>> print(r)
I am also a string.

\end{pyconcode}

\subsection{Insert Special Characters by Escaping Them \pythoninline{"\", \', \\"}}

Notice how we did not use a single quote in the second string because it was
enclosed (\emph{delimited}) by single quotes. The proper way to use a single
quote in a single quoted string or a double quote in a double quoted string
is to use escape the quote with a backslash:

\begin{pyconcode}
>>> s = "Previously, we said \"I'm a string.\"."
>>> print(s)
Previously, we said "I'm a string.".
>>> r = 'I\'m also a string.'
>>> print(r)
I'm also a string.

\end{pyconcode}

This is called \emph{escaping} a character\index{escaping
characters}. We \emph{escaped} the double quotes and single quote
respectively so that Python did not think it was the end of the
string.

You do not have to escape a single quote in a double quoted string and vice
versa:

\begin{pyconcode}
>>> s = "I can use single quotes '' here"
>>> print(s)
I can use single quotes '' here
>>> r = 'I can use double quotes "" here'
>>> print(r)
I can use double quotes "" here

\end{pyconcode}

If you want a string to go to a new line when printed, you use \pythoninline!\n! for that:

\begin{pyconcode}
>>> "Explicit is better than implicit.\nSimple is better than complex."
'Explicit is better than implicit.\nSimple is better than complex.'
>>> print("Explicit is better than implicit.\nSimple is better than complex.")
Explicit is better than implicit.
Simple is better than complex.

\end{pyconcode}

What, however, if you actually need a \pythoninline!\! in a string?
You can type two backslashes to escape the backslash:

\begin{pyconcode}
>>> print("C:\\Users\\hrivera")
C:\Users\hrivera

\end{pyconcode}

\subsection{Multi-Line Strings Using \pythoninline{"""}}

If you want strings to go on for two or more lines of Python code, you
have to enclose the string in three double quotes\index{multi-line
strings}. Everything between the two triple-quotes will be a part
of the string.

\begin{listing}[H]
  \vspace{-0.5em}
\begin{python3code}
weizsaecker = """We in the older generation owe to young people not the
fulfillment of dreams but honesty. We must help younger people to
understand why it is vital to keep memories alive. We want to help them
to accept historical truth soberly, not one-sidedly, without taking
refuge in Utopian doctrines, but also without moral arrogance. From our
own history we learn what man is capable of. For that reason we must not
imagine that we are quite different and have become better. There is no
ultimately achievable moral perfection. We have learned as human beings,
and as human beings we remain in danger. But we have the strength to
overcome such danger again and again."""
\end{python3code}
  \vspace{-1em}
  \caption{Excerpt of Richard von Weizs\"{a}cker's speech in
the Bundestag to commemorate the 40th anniversary of the end of World War II.}
  \vspace{-0.5em}
\end{listing}


\begin{listing}[H]
  \vspace{-0.5em}
\begin{pyconcode}
>>> r = """I hear America singing, the varied carols I hear,
... Those of mechanics, each one singing his as it should be blithe and strong,
... The carpenter singing his as he measures his plank or beam,"""
>>> print(r)
I hear America singing, the varied carols I hear,
Those of mechanics, each one singing his as it should be blithe and strong,
The carpenter singing his as he measures his plank or beam,

\end{pyconcode}
  \vspace{-1em}
  \caption{Excerpt of \emph{I Hear America Singing} by Walt Whitman.
    Notice the triple-dots in the Python interactive shell indicate that
    this is a multi-line segment of code.}
  \vspace{-0.5em}
\end{listing}


\subsection{Checking String Properties Using \pythoninline{len, in, ==, !=}}

Lists, tuples, and strings are \texttt{sequence} data
types\index{sequence data types} --- this means they support indexing,
slicing, and many of the functions that were introduced in the section
describing lists in lab 4.
A lot of other operations that work on lists also work on strings.
You can test membership of a string using \pythonindex{in}.

\begin{pyconcode}
>>> s = 'abcde'
>>> print('c' in s)
True
>>> print('f' in s)
False
>>> print('f' not in s)
True
>>> print('ab' in s)
True

\end{pyconcode}


You can measure the length of a string by writing \pythonindex{len()}.

\begin{pyconcode}
>>> s = 'abcDE'
>>> print(len(s))
5
>>> len("")
0
>>> len("Hello world!")
12

\end{pyconcode}


\subsubsection{Checking String-Specific Properties}

There are several functions that you can use to check if
a string contains certain sets or sequences of characters.
For example, you can check if a string is all uppercase using the
\pythoninline{s.isalpha()} function. These functions all start with
``is'' and return a boolean value.

\begin{pyconcode}
>>> s = 'abcDE'
>>> print(s.isnumeric())
False
>>> s.isalpha()
True
>>> s.islower()
False
>>> print(s.isupper())
False

\end{pyconcode}

Here is a complete list:

% generated with
% >>> [print('\n\item[\\pythonindex((s.{}))]  {}'.format(s.__name__, s.__doc__.split('\n\n')[1]))
% ...  for s in [str.isalnum, str.isdecimal, str.isidentifier, str.isnumeric, str.isspace, str.isupper,
% ...            str.isalpha, str.isdigit, str.islower, str.isprintable, str.istitle]]

\begin{description}
\item[\pythonindex{s.isalnum()}]  Return True if all characters in S are alphanumeric
  and there is at least one character in S, False otherwise.

\item[\pythonindex{s.isdecimal()}]  Return True if there are only decimal characters in S,
  False otherwise.

\item[\pythonindex{s.isidentifier()}]  Return True if S is a valid identifier according
  to the language definition.

\item[\pythonindex{s.isnumeric()}]  Return True if there are only numeric characters in S,
  False otherwise.

\item[\pythonindex{s.isspace()}]  Return True if all characters in S are whitespace
  and there is at least one character in S, False otherwise.

\item[\pythonindex{s.isupper()}]  Return True if all cased characters in S are uppercase and there is
  at least one cased character in S, False otherwise.

\item[\pythonindex{s.isalpha()}]  Return True if all characters in S are alphabetic
  and there is at least one character in S, False otherwise.

\item[\pythonindex{s.isdigit()}]  Return True if all characters in S are digits
  and there is at least one character in S, False otherwise.

\item[\pythonindex{s.islower()}]  Return True if all cased characters in S are lowercase and there is
  at least one cased character in S, False otherwise.

\item[\pythonindex{s.isprintable()}]  Return True if all characters in S are considered
  printable in repr() or S is empty, False otherwise.

\item[\pythonindex{s.istitle()}]  Return True if S is a titlecased string and there is at least one
  character in S, i.e. upper- and titlecase characters may only
  follow uncased characters and lowercase characters only cased ones.
\end{description}


\subsection{Indexing and Slicing Strings}

A character is a single letter, digit, space, newline, or other symbol.
Think of a string as a list of characters.
This is because a string can be indexed like a list in order to pick out a specific \emph{character}.
For example:

\begin{pyconcode}
>>> s = "99 Python Problems"
>>> s[0]
'9'
>>> s[2]
' '
>>> s[3]
'P'
>>> s[-1]
's'
>>> s[100]
Traceback (most recent call last):
  File "stdin", line 1, in <module>
    s[100]
IndexError: string index out of range

\end{pyconcode}

Slicing\index{slicing strings} works on strings. Review lab 4 for a
description of several types of slices.

\begin{pyconcode}
>>> s = "The cat in the hat"
>>> print(s[2])
e
>>> print(s[4:7])
cat
>>> print(s[15:18])
hat
>>> print(s[14:18])
 hat
>>> print(s[17:14:-1])
tah
>>> print(s[17::-1])
tah eht ni tac ehT

\end{pyconcode}

\subsection{Creating New Strings}

Unlike lists, strings are \emph{immutable}\index{immutable}. A
string's components can be accessed but can't be
modified.

\begin{pyconcode}
>>> s = 'Python'
>>> s[0] = 'J'
Traceback (most recent call last):
  File "<stdin>", line 1, in <module>
TypeError: 'str' object does not support item assignment

\end{pyconcode}

In this case, Python raises a \pythonindex{TypeError}, just like when
attempting to modify tuples.
If you want to change a string, you must create a new string:

\begin{pyconcode}
>>> s = 'Python'
>>> r = 'J' + s[1:]
>>> print(r)
Jython

\end{pyconcode}


\subsubsection{Concatenation and Repetition Using \pythoninline{+, *}}

You can concatenate, or join, two strings using the \pythonindex{+}
operator; this operation creates a new string.

\begin{pyconcode}
>>> r = "The cat"
>>> s = " in the hat"
>>> r + s
'The cat in the hat'
>>> s + r
' in the hatThe cat'

\end{pyconcode}

In addition to that, we can repeat strings using the multiplication operator
\pythonindex{*}:

\begin{pyconcode}
>>> s = "Hi"
>>> r = 5 * s
>>> print(r)
HiHiHiHiHi
>>> n = 3
>>> print(r + 'cat' * n)
HiHiHiHiHicatcatcat

\end{pyconcode}


\subsubsection{Replacing Substrings With \pythoninline{string.replace(old, new)}}

The \pythonindex{string.replace(old, new)} replaces the substring
\pythoninline{old} with the string \pythoninline{new}.
For example:

\begin{pyconcode}
>>> s = 'abcDE'
>>> r = s.replace('ab', 'more')
>>> print(r)
morecDE
>>> print(s)
abcDE

\end{pyconcode}

\subsubsection{Creating Uppercase and Lowercase Strings}

The functions \pythonindex{s.upper()} \pythonindex{s.lower()}, and \pythonindex{s.capitalize()}
create new strings that are entirely uppercase or lowercase, or capitalized.

\begin{pyconcode}
>>> s = 'abcDE'
>>> print(s.upper())
ABCDE
>>> s.lower()
'abcde'
>>> s.capitalize()
'Abcde'

\end{pyconcode}

\subsection{Splitting a String Into a List of Strings Using \pythoninline{str.split()}}

How can a string of space separated words be transformed into a
list of words? For example, the string \pythoninline{"The cat in the hat"}
might be more useful as a list of strings:
\pythoninline{["The", "cat", "in", "the", "hat"]}.
Python's \pythonindex{str.split(delimiter)} function takes a string and returns
a list of all substrings that were separated by a given delimiter.

To split words separated by spaces, the delimiter would be the string \pythoninline{" "}.
Here are some examples:

\begin{pyconcode}
>>> "The cat in the hat".split(" ")
['The', 'cat', 'in', 'the', 'hat']
>>> "1,2,3,4,5,6,7".split(",")
['1', '2', '3', '4', '5', '6', '7']
>>> "1, 2, 3, 4, 5,  6,7".split(", ")  # not magic
['1', '2', '3', '4', '5', ' 6,7']
>>> "none_here".split(" ")
['none_here']

\end{pyconcode}

To split a string on any kind of whitespace, such as spaces or newlines, call the
function without any arguments: \pythoninline{str.split()}.
\begin{pyconcode}
>>> "The cat in\nthe     hat".split(" ")
['The', 'cat', 'in\nthe', '', '', '', '', 'hat']
>>> "The cat in\nthe     hat".split()
['The', 'cat', 'in', 'the', 'hat']

\end{pyconcode}

\section{Documenting Your Functions and Modules With Docstrings}

The \pythoninline{help()} function introduced in the previous labs
prints the given function's documentation which is known as a
docstring. You can write detailed documentation for your functions by
placing a triple-quoted string immediately under the beginning of a
function definition (after the \pythoninline{def f(...):}).

The docstring is a comment describing the use of a function: the
documentation should include information about each argument, the
return value, and the function's behavior.

Function comments must follow the Python Enhancement Proposal 257 (PEP
257), a document that describes Python docstring conventions. It is
found at:

\begin{center}
  \url{https://www.python.org/dev/peps/pep-0257/}
\end{center}

The highlights of PEP 257 are listed below.

For short functions, the docstring may be a one-liner:

\begin{python3code}
def midpoint(a, b):
    """Find and return the midpoint of the given a and b."""
    return (a+b)/2
\end{python3code}

For larger functions or for a longer explanation, you must write a
short summary followed by as much text as needed:

\begin{python3code}
def calculate_weekly_pay(pay_rate, hours, tax_rate):
    """Find net weekly pay after taxes with overtime set to 1.5 the pay rate.

    The net pay after taxes is calculated given the number of hours worked in a
    week, a pay rate, and a flat tax rate. Takes into consideration overtime pay
    at 1.5 times the pay rate.

    Arguments:
    pay_rate -- rate of pay
    hours -- number of hours worked in one week
    tax_rate -- flat tax rate (for example, 0.15 for 15%)
    """
    pay_before_taxes = hours * pay_rate

    # Add overtime payment if necessary
    if hours > 40:
        pay_before_taxes += (hours - 40) * pay_rate * 0.5

    pay_after_taxes = pay_before_taxes * (1 - tax_rate)
    return pay_after_taxes
\end{python3code}

\begin{warningbox}{Docstrings are Required}
From now on, you will be required to put a \emph{docstring} at the beginning
of every function that you write.
\end{warningbox}


\section{Optional Function Arguments: Keyword Arguments}

Functions may take zero or more required arguments, these
are called \textsl{positional arguments}\index{positional arguments}.
There is a special syntax that can be used to define \textsl{optional
arguments}\index{optional arguments}.
For example, the following function takes either one or two arguments. The
second argument has a default value of 5.

\begin{pyconcode}
>>> def f(fst, snd = 5):
...    return fst * snd
>>> f(10)
50
>>> f(10, 6)
60
>>> f()
Traceback (most recent call last):
  File ``<stdin>'', line 1, in <module>
TypeError: f() missing 1 required positional argument: 'fst'

\end{pyconcode}

An important feature is that keyword arguments may be labeled
and listed in any order when calling the function. For example,
the following function takes two keyword arguments. There are five
valid ways to call this function. You may pass no arguments, then
the default values are used. You may pass either argument by itself
or you may pass both arguments in any order.

\begin{pyconcode}
>>> def p(greeting='Hello', subject='world'):
...     return '{} {}'.format(greeting, subject)
>>> p()
'Hello world'
>>> p(greeting='Goodbye')
'Goodbye world'
>>> p(subject='humans')
'Hello humans'
>>> p(greeting='Bonjour', subject='le monde')
'Bonjour le monde'
>>> p(subject='le monde', greeting='Bonjour')
'Bonjour le monde'

\end{pyconcode}


% \section{Sample Program} % TODO


\newpage
\section{Exercises}\label{exercises}
\begin{ex}[stringfun.py]
  \emph{This exercise is a modification of exercises in the Google Python
    class, licensed under the
    \href{http://www.apache.org/licenses/LICENSE-2.0.html}{Apache License 2.0}.}

The following functions should not print anything.
Create a user interface that uses the functions, the \pythoninline{input()}
function, and the \pythoninline{print()} function.

\begin{description}
  \item[\pythoninline{ends()}]
    Write a function that takes in a string from the user and
    prints just the first two and the last two characters of the string.
    You may assume that any input will be at least 2 characters long.

    \begin{verbatimcode}
      == ends ==
      Enter a string >>> spring
      spng
    \end{verbatimcode}

  \item[\pythoninline{mix()}]
    Write a function that takes two strings \pythoninline!a! and
    \pythoninline!b! and prints the two strings concatenated, but with the first
    two characters of each word swapped with the other word's first two
    characters. You may assume that any input will be at least two characters
    long.
    For example:

    \begin{verbatimcode}
      == mix ==
      String a >>> german
      String b >>> english
      enrman geglish
    \end{verbatimcode}

    \begin{verbatimcode}
      == mix ==
      String a >>> dog
      String b >>> dinner
      dig donner
    \end{verbatimcode}

%   \item[\pythoninline{splitit()}]
%     Consider dividing a string into two halves. If the length
%     is even, the front and back halves are the same length. If the length is odd,
%     we'll say that the extra character goes in the front.
% 
%     For example, in \pythoninline!'abcde'!, the front half is \pythoninline!'abc'!
%     and the back half is \pythoninline!'de'!.
% 
%     Write a function that takes in two strings \pythoninline!a! and
%     \pythoninline!b!  and prints a-front + b-front + a-back + b-back.
% 
%     For example,
% 
%     \begin{verbatimcode}
%       == splitit ==
%       String a >>> abcd
%       String b >>> efghi
%       abefgcdhi
%     \end{verbatimcode}
% 
%     \begin{verbatimcode}
%       == splitit ==
%       String a >>> this dinner is
%       String b >>> what am i doing1
%       this diwhat am nner isi doing1
%     \end{verbatimcode}
\end{description}
\end{ex}

\begin{ex}[luhns.py]
Luhn's algorithm provides a quick way to
 check if a credit card is valid or not. The algorithm consists of four
 steps.

 \begin{enumerate}

\item We will use the Diners Club card number 38520000023237 as an example.

 \begin{center}
 \begin{tabular}{llllllllllllll}
3 & 8 & 5 & 2 & 0 & 0 & 0 & 0 & 0 & 2 & 3 & 2 & 3 & 7\\
 \end{tabular}
 \end{center}

\item Starting with the second to last digit (ten's column digit),
     multiply every other digit by two.
 \begin{center}
 \begin{tabular}{llllllllllllll}
6 & 8 & 10 & 2 & 0 & 0 & 0 & 0 & 0 & 2 & 6 & 2 & 6 & 7
 \end{tabular}
 \end{center}

\item Sum all the digits of the resulting number.
 Note that for 10, you also sum its digits: $(1 + 0)$; for an 18, you would do $1 + 8$.
 \[ 6 + 8 + (1 + 0) + 2 + 0 + 0 + 0 + 0 + 0 + 2 + 6 + 2 + 6 + 7 = 40 \]

 \item If the total sum modulo by 10 is zero, then the card is valid;
     otherwise it is invalid.
 For this example, the last step is to check if $40$ modulus 10 is equal to 0, which is true.
 So the card is valid.
 \end{enumerate}

 Write a program that implements Luhn's Algorithm for validating credit
 cards. It should ask the user to enter a credit card number and tell the
 user whether it is valid or not.

 \emph{There must be a separate function called \texttt{validate} that
  takes in a card number and validates it. It should not print anything
  nor accept user input.}

\end{ex}

\begin{infobox}{Testing Script}
You can test your code in \texttt{luhns.py} by running the test script
\texttt{test\_luhns.py}. This will automatically test your function on various
card numbers. The output of the test script may look like:

\begin{bashcode}
$ ls
luhns.py test_luhns.py ...
$ python3 test_luhns.py
luhns.validate("49927398717") should return False, but returned True.
luhns.validate("1234567812345678") should return False, but returned True.
The function luhns.validate correctly verified 3 out of 5 card numbers.
\end{bashcode}

\end{infobox}

% make sure students understand the tuple-encoding
\begin{ex}[fractions.py] Any fraction can be written as the division
of two integers. You could express this in Python as a tuple ---
\pythoninline{(numerator, denominator)}.

  For example, the fractions $\frac{1}{2}$, $\frac{10}{7}$, and
  $\frac{499}{10001}$ can be represented using the tuples
  \pythoninline{(1, 2)}, \pythoninline{(10, 7)}, and \pythoninline{(499, 10001)}.

  Write the following functions: 

  \begin{description}
    \item[\pythoninline{reduce(fraction)}] This function takes a fraction, reduces it, and returns the result.
        For example, \pythoninline{reduce((8, 4))} should return \pythoninline{(2, 1)}.
        To reduce a fraction $a/b$, divide $a$ and $b$ by their GCD.
        The result is $(a/d) / (b/d)$.
        The math module comes with the \pythoninline{math.gcd} function.
    \item[\pythoninline{add(fraction1, fraction2)}] Given two fractions as tuples, add them.
    \item[\pythoninline{multiply(fraction1, fraction2)}] Given two fractions as tuples, multiply them.
    \item[\pythoninline{divide(fraction1, fraction2)}] Given two fractions as tuples, divide them.
  \end{description}
  These functions should not use \pythoninline{input()} or \pythoninline{print()}. 

  Write a small command-line interface such that the user running your
  script sees something like this:

  \begin{bashcode}
$ python3 fractions.py
Enter a fraction >>> 5/3
Enter a fraction >>> 10/3
Reduced fractions to 5/3 and 10/3.
Sum of the fractions: 3/1.
Multiplication of the fractions: 50/9.
Division of the first by the second: 1/2 .
$ python3 fractions.py
Enter a fraction >>> 3628800/479001600 
Enter a fraction >>> 10/1000
Reduced fractions to 1/132 and 1/100 
Sum of the fractions: 29/1650 
Multiplication of the fractions: 1/13200 
Division of the first by the second: 25/33 

  \end{bashcode}

  The \pythoninline{split} function may be useful when converting strings ``xxx/yyy'' into 
  tuples (xxx, yyy).
  You should use the functions \pythoninline{add, multiply, reduce}, and
  \pythoninline{divide} from your main function.
\end{ex}


%\newpage
%  ---------------- unfinished -----------------------
%\appendix\section{Managing 135 Alphabets: Unicode and String Comparisons} TODO
%% 128,000 characters covering 135 modern and historic scripts, as well as multiple symbol sets.
%
%% emoji, first script
%\subsection{String Representation}
%% a = 96
%% 8-characters
%
%\subsection{Unicode}\label{unicode}
%
%  Multi-lingual writing.
%  (\url{https://docs.python.org/3.5/howto/unicode.html}) Apple II BASIC
%  program written by a French speaker might have lines like these:
%
%\begin{verbatim}
%PRINT "MISE A JOUR TERMINEE"
%PRINT "PARAMETRES ENREGISTRES"
%\end{verbatim}
%
%  Those messages should contain accents (terminée, paramètre,
%  enregistrés) and they just look wrong to someone who can read French.
%  Python's strings can contain letters with accents and almost every
%  known glyph on earth because they use the Unicode encoding\index{Unicode encoding}.
%  The major difference between Python 2 and Python 3 is unicode.
%
%\subsection{String Comparison}
%
%  Strings are strings of characters, characters are integers.
%  Example using ord() function.
%  Strings are compared by their code points, not by abstract characters.
%  (\url{https://docs.python.org/3.5/library/unicodedata.html}) For
%  example, the character U+00C7 (LATIN CAPITAL LETTER C WITH CEDILLA)
%  can also be expressed as the sequence U+0043 (LATIN CAPITAL LETTER C)
%  U+0327 (COMBINING CEDILLA).
%
%  If you're comparing multi-lingual strings, import the unicode module with
%  \pythonindex{import unicode}
%  and then use \pythonindex{unicode.normalize()}.
%  For example:

\newpage

\printindex
\vfill
\section*{Submitting}\label{submitting}
\addcontentsline{toc}{section}{Submitting}

You should submit your code as a tarball that contains all the
exercise files for this lab. The submitted file should be named
\begin{center}
  \texttt{cse107\_firstname\_lastname\_lab\thelabnumber.tar.gz}
\end{center}

\begin{center}
  \textbf{Upload your tarball to Canvas.}
\end{center}

\listofexercises

Exercises start on page~\pageref{exercises}.

\end{document}
