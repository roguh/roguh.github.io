\documentclass[11pt]{cselabheader}

\newcommand{\thelabnumber}{4}
\newcommand{\thetitle}{Lists and For Loops}
\newcommand{\theauthor}{CSE/IT 107L}

% Write title and author
\title{\vspace{-4em} \thetitle}  % reduce space before the title
\author{\theauthor}
\date{NMT Department of Computer Science and Engineering}

\fancyhead[R]{Lab \thelabnumber: \thetitle}
\fancyhead[L]{\theauthor}

\makehyperref
\makeindex[title=Index of New Functions and Methods, intoc]

\begin{document}

\pagenumbering{roman}
\maketitle
\hrule

\begin{quotation}
``Programming is learned by writing programs.''
\end{quotation}
\begin{flushright}
--- Brian Kernighan
\end{flushright}

\begin{quotation}
``The purpose of computing is insight, not numbers.''
\end{quotation}
\begin{flushright}
--- Richard Hamming, 1962
\end{flushright}

\begin{quotation}
  ``From our own history we learn what man is capable of. For that reason we
  must not imagine that we are quite different and have become better.''
\end{quotation}
\begin{flushright}
  --- Richard von Weizs\'{a}cker
%\\\textit{Speech to commemorate end of WWII, May 8 1985}
\end{flushright}

\begin{figure}[H]
  \centering
  \includegraphics[width=0.88\textwidth]{img/xkcd_1185.png}
  \caption{\url{http://xkcd.com/1185}}
\end{figure}

\newpage

\section*{Introduction}

We have discussed several types of Python values. These
include integers, floating point numbers, strings, and booleans.  This
lab marks the start of your introduction containers for Python
values. These are lists, tuples, strings, sets, and dictionaries.
They are used to organize Python values and are
called \emph{data structures}.
In this lab, you will learn about lists, an ordered collection of
Python values such as \pythoninline{[1, 2, 3]}. You will learn how to
create, modify, and slice lists. You
will also learn more about \pythoninline{for} loops, a
powerful Python feature for accessing values in data structures.

\vspace{-1.5em}
\tableofcontents

\newpage
\pagenumbering{arabic}

\section{Lists: Ordered Collections of Python Values}

One of the most important data types in Python is the list\index{list}. A list is an ordered
collection of Python values. For example, we might create a list of numbers
representing data points on a graph or create a list of strings representing
names of students in a class. To create a Python list value, we simply
place comma-separated values inside square brackets.
The values within a list are called elements\index{list elements}.
The number of values within a list is called the length\index{length} of a list.

\begin{python3code}
python_list = [value1, value2, value3, ...]
\end{python3code}

For example, the following list assigned to the variable \pythoninline{values}
contains the integers 1, 2, 3, and 4; they're in this order.

\begin{pyconcode}
>>> values = [10, 20, 30, 40, 50]
>>> print(values)
[10, 20, 30, 40, 50]

\end{pyconcode}

Here is a list of strings:

\begin{pyconcode}
>>> strings = ['Mercury', 'Pluto', 'Luna']
>>> strings
['Mercury', 'Pluto', 'Luna']

\end{pyconcode}

How do you print the strings within the list? How do you modify the
elements of a list? How do you combine two lists? Can you delete an element?
As we will see, you can perform many operations on their elements or on the list itself.


\subsection{Nested Lists: Lists Containing Lists}

A Python list is just another type of Python value.
Lists can store any type of Python value, therefore you
may create lists that contain lists, also known as
nested lists \index{nested lists}.

For example, this is a 2 dimensional list that represents a $4\times 5$
2-color picture:

\begin{pyconcode}
>>> picture = [[0, 1, 0, 1, 0],
...            [0, 0, 0, 0, 0],
...            [1, 0, 0, 0, 1],
...            [0, 1, 1, 1, 0]]
>>> print(picture)
[[0, 1, 0, 1, 0], [0, 0, 0, 0, 0], [1, 0, 0, 0, 1], [0, 1, 1, 1, 0]]

\end{pyconcode}

The triple-dots \bashindex{...} under the \bashinline{>>>} indicate a
multi-line code entry in the Python interactive shell. Use
Shift+Enter to type multiple lines.

This is a 2 dimensional list of strings. Notice that sublists do not have
to be of the same length.

\begin{pyconcode}
>>> systems = [['Venus'], ['Earth', 'Moon'], ['Mars', 'Phobos', 'Deimos']]
>>> print(systems)
[['Venus'], ['Earth', 'Moon'], ['Mars', 'Phobos', 'Deimos']]

\end{pyconcode}

Finally, note that you can nest lists as deep as needed:
\begin{pyconcode}
>>> balancedleaves = [[[1, 3], [4, 6]],
...                   [[11, 10, 8], [[15, 17, 41], [50], [55, 52, 53]]]]

\end{pyconcode}


\subsection{Accessing Elements Using Indexing}

To access\index{element access} (extract or look at) an element within a list, you may
use indexing. To index\index{indexing} % TERMINOLOGY
the n$^\text{th}$ element in a list, write the list or a variable
containing the list and then write square brackets with the number
$n - 1$. This is because lists are 0-indexed\index{0-indexing}, meaning
the first element of a list is at index 0 and the last element is at
index N-1 where N is the number of elements in the list.
For example: the first year in a decade ends in 0, but the last year ends in 9.
Here's a picture that shows two 0-indexed lists:

\begin{figure}[H]
  \begin{center}
  \includegraphics[width=\linewidth]{img/pythontutor_lists.png}
  \end{center}
  \caption{Two lists visualized using pythontutor.com. One is a list of integers, the other list contains strings. Note, each element has its index written directly above it. The first element in each list has index 0.}
\end{figure}

In these samples, we take values out of previously introduced lists:

\begin{pyconcode}
>>> values = [10, 20, 30, 40, 50]
>>> values[0]
10
>>> values[3]
40
>>> values[5]  # does not exist
Traceback (most recent call last):
  File '<stdin>', line 1, in <module>
    values[5]  # does not exist
IndexError: list index out of range
>>> print('the first and last elements are {} and {}'.format(values[0], values[4]))
the first and last elements are 10 and 50
>>> strings = ['Mercury', 'Pluto', 'Luna']
>>> strings[0]
'Mercury'
>>> strings[2]
'Luna'

\end{pyconcode}

Python lists contain Python values, so we can assign the elements of a list
to a variable. For example,

\begin{pyconcode}
>>> values = [1, 2, 3, 4]
>>> first = values[0]
>>> print(first)
1

\end{pyconcode}

\subsubsection{Access Elements From the End Using Negative Indices}
If we wish, we can use negative array indices to reference elements starting at
the end of the list. For example, -1 is the last element, -2 is the second to
last element, and so on.

\begin{pyconcode}
>>> values = [32, 1, 54, -3, 6]
>>> print(values[-1])
6
>>> print(values[-2])
-3

\end{pyconcode}

\subsubsection{A Potential Error: Index Out of Range}

What happens if you try to get the element at index $5$ in a
list of 5 elements?

\begin{pyconcode}
>>> values = [10, 20, 30, 40, 50]
>>> values[5]
Traceback (most recent call last):
  File ``<stdin>'', line 1, in <module>
IndexError: list index out of range

\end{pyconcode}

Python will raise an \pythonindex{IndexError} with the message
``list index out of range''. To avoid such an error, check the length
of every list with an unknown length by using the \pythoninline{len}
function before you try to access its elements.

This error also occurs when using negative indexing and the index is
too small; such as when trying to access the $-6^\text{th}$ element in
a list of 5 elements.

\subsection{\texorpdfstring%
  {Checking List Properties With \pythoninline{len, sum}}
  {Checking List Properties With len, sum}}
The \pythoninline{len(l)} function returns the length of the list passed to it.

\begin{pyconcode}
>>> values = [1, 2, 3, 12, 123]
>>> print(len(values))
5
>>> empty = []
>>> print(len(empty))
0

\end{pyconcode}

The \pythoninline{sum(l)} function sums of every value in a
list, so long as every value in the list is a number. If any values
are not numbers, an error will occur.

\begin{pyconcode}
>>> values = [1, 2, 3, 12, 123]
>>> print(sum(values))
141
>>> values.append('test')
>>> print(sum(values))
Traceback (most recent call last):
  File '<stdin>', line 1, in <module>
TypeError: unsupported operand type(s) for +: 'int' and 'str'

\end{pyconcode}


\subsubsection{Lists to Booleans Using \pythoninline{in, ==, !=}}

A common operation is to test if a list contains a given value. We can do this
using the \pythonindex{in} keyword. We can also test if a list does not contain a
value using \pythonindex{not in}.

\begin{pyconcode}
>>> values = [1, 'test', 30, 20]
>>> print(1 in values)
True
>>> print('test' in values)
True
>>> print(2 in values)
False
>>> print(2 not in values)
True

\end{pyconcode}

This could be used to simplify the example from Lab 2 involving checking user
input against multiple valid passwords. Lists make it easier to add or remove
passwords.

\begin{python3code}
passwords = ['hunter2', 'hunter3', 'hunter4']

user_in = input('Please enter your password: ')

if user_in in passwords:
    print('Correct password. Welcome!')
else:
    print('Incorrect password.')
\end{python3code}



\subsection{Modifying Lists and Their Elements}
The elements of a list are similar to variables.
Think of each index as a ``slot'' where a Python value can be stored.
Because of this, we can reassign the elements of a list by
using the assignment operator\index{assignment operator, =}.

\begin{pyconcode}
>>> values = [1, 2, 3]
>>> values[0] = 16
>>> values
[16, 2, 3]
>>> values[2] = 1
>>> values
[16, 2, 1]
>>> values[3] = 10
Traceback (most recent call last):
  File "<stdin>", line 1, in <module>
IndexError: list assignment index out of range

\end{pyconcode}

\subsubsection{Add an Element Using \pythoninline{list.append(value)}}

Many functions exist to help manipulate lists. One of these is called
\pythonindex{list.append(value)}. This adds a new value \pythoninline{value} onto
the end of the list \pythoninline{list}. Do not name any of your variables \pythoninline{list},
this name has special significance in Python.

\begin{pyconcode}
>>> values = [1, 2, 3]
>>> values.append(12)
>>> values.append(123)
>>> print(values)
[1, 2, 3, 12, 123]

\end{pyconcode}

\subsubsection{Add an Element to Any Index Using \pythoninline{list.insert(index, value)}}

\pythonindex{list.insert(index, value)} also inserts a value, but it allows you to choose where it
goes. The first argument of the function is the index you want your new value to
be located. Other values will be moved to make room.

\begin{pyconcode}
>>> values = [-1, 0, 1]
>>> values.insert(2, 10)
>>> print(values)
[-1, 0, 10, 1]

\end{pyconcode}

\subsubsection{Remove Elements Using \pythoninline{list.pop()} and \pythoninline{list.pop(index)}}

The \pythonindex{list.pop()} function is returns the Python
value at the end of a list, but it also removes the value from the list.
If given a value, then \pythonindex{list.pop(index)} will remove the
value at that index.

\begin{pyconcode}
>>> values = [1, 2, 3, 12, 123]
>>> print(values.pop())
123
>>> values
[1, 2, 3, 12]
>>> print(values.pop(0))
1
>>> print(values)
[2, 3, 12]

\end{pyconcode}


\subsubsection{Joining Two Lists With \pythoninline{+}}

Two lists can be merged (joined or concatenated) together in order to
combine them into one list using the \pythonindex{+} operator.

\begin{pyconcode}
>>> values = [1, 2, 3]
>>> more_values = [3, 2, 111]
>>> combined = values + more_values
>>> print(combined)
[1, 2, 3, 3, 2, 111]

\end{pyconcode}

The \pythoninline{+} operator preserves the order of the two lists.

\subsubsection{Sorting and Reversing Lists}

To sort a list, use the \pythonindex{list.sort()} function or the
\pythonindex{sorted(l)} function. Both functions sort lists of strings
in dictionary-order, and lists of numbers in ascending order.

The keyword argument \pythoninline{reverse} can be set to
\pythoninline{False} to sort the list in descending order.
This looks like: \pythonindex{list.sort(reverse=False)}.
We will discuss keyword arguments in later labs.

\begin{pyconcode}
>>> values = [10, 20, 30, 40, 50]
>>> values.sort(reverse=True)
>>> values
[50, 40, 30, 20, 10]
>>> values.sort()
>>> values
[10, 20, 30, 40, 50]

\end{pyconcode}

The \pythonindex{list.reverse()} function reverses the given list.

\begin{pyconcode}
>>> strings = ['Mercury', 'Pluto', 'Luna']
>>> strings.reverse()
>>> strings
['Luna', 'Pluto', 'Mercury']

\end{pyconcode}

\section{Accessing Sublists by Slicing Lists}
A sublist contains all of the elements of another list and no other elements.
For example, \pythoninline{[1,2,3]} is a sublist of \pythoninline{[1,2,3,4,5]}.
We can access many different sublists from a Python list by
using slicing. Four of slicing methods are shown in the following
subsections.

A list is a sublist of itself, and the empty list \pythonindex{[]} is
a sublist of every list. This means slices will always result in
either the original list itself, the empty list, or something between
the two extremes.

Everything that can be done with a slice can also be done using
for-loops or while-loops; but slices are a useful Python feature
for writing concise code.

\subsection{\texorpdfstring%
  {Access All Elements After $M$ Using \pythoninline{[M:]}}
  {Access All Elements After M Using {[}M:{]}}}

The slice of the list \pythoninline{l}, \pythonindex{l[m:]},
will contain every element after the index \pythoninline{m}.
Note that \pythoninline{l[m:]} includes \pythoninline{l[m]}.
For example,

\begin{pyconcode}
>>> strings = ['Mercury', 'Venus', 'Earth', 'Mars']
>>> strings[2:]
['Earth', 'Mars']
>>> strings[0:]
['Mercury', 'Venus', 'Earth', 'Mars']

\end{pyconcode}

\subsection{\texorpdfstring%
  {Access All Elements Before $N$ Using \pythoninline{[:N]}}
  {Access All Elements Before N Using {[}:N{]}}}
Similarly, the slice \pythonindex{l[:n]} will create a sublist
that contains every element before the index \pythoninline{n}.
Note that \pythoninline{l[:n]} does not include \pythoninline{l[n]}.

\begin{pyconcode}
>>> strings = ['Mercury', 'Venus', 'Earth', 'Mars']
>>> strings[:0]
[]
>>> strings[:1]
['Mercury']
>>> strings[:3]
['Mercury', 'Venus', 'Earth']

\end{pyconcode}

If out-of-bounds indices are used, the slice returns
a list instead of raising an exception.

\begin{pyconcode}
>>> values = [10, 20, 30, 40, 50]
>>> values[:100000]
[10, 20, 30, 40, 50]
>>> values[100000:]
[]

\end{pyconcode}

\subsection{\texorpdfstring%
  {Access All Elements Between $M$ and $N$ Using \pythoninline{[M:N]}}
  {Access All Elements Between M and N Using {[}M:N{]}}}
To access all elements between two indices \pythoninline{m} and
\pythoninline{n}, slice the list using \pythonindex{l[m:n]}.
Here are some examples:

% Odd numbers with prime number of divisors
\begin{pyconcode}
>>> values = [3, 5, 7, 9, 11, 13, 17, 19, 23, 25, 29, 31]
>>> values[1:5]
[5, 7, 9, 11]
>>> values[100:115]
[]
>>> values[-4:-10]
[]
>>> values[-10:-4]
[7, 9, 11, 13, 17, 19]

\end{pyconcode}

\subsection{Skip Every $S$ Elements Using \pythoninline{[M:N:S]},
  \pythoninline{[:N:S]}, \pythoninline{[M::S]}, and \pythoninline{[::S]}}

To skip elements, you can use a third index \pythoninline{s}
that defines the step size. To skip every other element between two
indices, use the slice \pythoninline{l[m:n:2]}.
To skip every \pythoninline{s} element, use this slice: \pythonindex{l[m:n:s]}.
This Python interactive shell session demonstrates the concept:

\begin{pyconcode}
>>> values = [3, 6, 9, 12, 15, 18, 21, 24, 27, 30]
>>> values[0:9:2]
[3, 9, 15, 21, 27]
>>> values[1:10:2]
[6, 12, 18, 24, 30]
>>> values[1:10:4]
[6, 18, 30]
>>> values[0:10:4]
[3, 15, 27]
>>> values[0:10:3]
[3, 12, 21, 30]

\end{pyconcode}

You can omit either the first or second index to include either every
element starting from the beginning or from the end.  For example,

\begin{pyconcode}
>>> values = [3, 6, 9, 12, 15, 18, 21, 24, 27, 30]
>>> values[:9:2]
[3, 9, 15, 21, 27]
>>> values[1::4]
[6, 18, 30]
>>> values[::4]
[3, 15, 27]

\end{pyconcode}


\section{Lists are Mutable: Side-Effects of Modifying Elements}

What does this program print?

\begin{python3code}
x = [4, -4]
y = x
x[0] = x[0] + 1
print y
print x
\end{python3code}

Both variables \pythoninline{x} and \pythoninline{y} store the same
list, not a copy of the list. Therefore, when you modify one variable,
they both change.

\begin{figure}[H]
  \begin{center}
  \includegraphics[width=0.5\linewidth]{img/pythontutor_listvalues.png}
  \end{center}
  \caption{A visualization of the result of modifying a list when two variables are
  storing it. Try to run the code.}
\end{figure}

\section{\texorpdfstring%
  {Traversing Lists Using \pythoninline{for} Loops}
  {Traversing Lists Using For Loops}}

In Lab 1, we saw how to use \pythoninline{for} loops.
These were only the most basic type of \pythoninline{for}
loop. In fact, Python has many methods for traversing (or iterating)
through data in a controlled way, thus avoiding the potentially
unpredictable behavior of \pythoninline{while} loops. Most of
those methods involve a \pythonindex{for} loop.
This is how you access each element in a list:

\begin{python3code}
values = [1, 2, 3, 4]
for value in values:
    # code that uses the `value`
\end{python3code}

Suppose you have a list of strings called \pythoninline{strings} that
you need to print.  You could write a while loop to print all
\pythoninline{len(strings)} elements, or you could use this for-loop:

\begin{python3code}
lines = ['# arithmetic.py', 'x = 7', 'y = 5 + x',
         'print(y)', 'print(x)']
for line in lines:
    print(line)
\end{python3code}

Like any other loop, you can have any type of Python code
under the indentation. For example, you could check the current element
at every iteration to see if it is larger than the largest known element
and in this way you can find the maximum element in the list.

\begin{python3code}
elements = [10, 14, -3, 5, -100, 12, 4]
maximum = elements[0]  # first guess
for element in elements:
    if element > maximum:
        maximum = element
\end{python3code}

If you have a nested list, you can traverse it using nested
for loops:

\begin{python3code}
picture = [[0, 1, 0, 1, 0],
           [0, 0, 0, 0, 0],
           [1, 0, 0, 0, 1],
           [0, 1, 1, 1, 0]]
for row in picture:
    for bit in row:
        if bit != 0:
            print(bit, end='')  # print without the newline
    print()  # print newline
\end{python3code}


\subsection{\texorpdfstring%
  {Access the Index of Current Elements With \pythoninline{enumerate()}}
  {Access the Index of Current Elements With enumerate()}}

If you have a list and need to access the index and the element at the
index at the same time, you should consider the \pythonindex{enumerate(l)}
function. This example prints the elements in a list, their index,
and the next element in the list (if it exists).

\begin{pyconcode}
>>> values = [10, 20, 30, 40, 50]
>>> for index, element in enumerate(values):
...     if index < len(values) - 1:
...         next_element = values[index + 1]
...         print('{} at {}, next: {}'.format(element, index, next_element))
...     else:
...         print('{} at {}'.format(element, index))
10 at 0, next: 20
20 at 1, next: 30
30 at 2, next: 40
40 at 3, next: 50
50 at 4

\end{pyconcode}

As another example, suppose you have a list of strings that represent lines of
Python code. You can print each string and the line number by using this code:

\begin{pyconcode}
>>> lines = ['# arithmetic.py', 'x = 7', 'y = 5 + x',
...          'print(y)', 'print(x)']
>>> for index, line in enumerate(lines):
...     print('{}: {}'.format(index + 1, line))
1: # arithmetic.py
2: x = 7
3: y = 5 + x
4: print(y)
5: print(x)

\end{pyconcode}

\subsection{\texorpdfstring%
  {Traverse Multiple Lists Using \pythoninline{zip()}}
  {Traverse Multiple Lists Using zip()}}

The \pythonindex{zip(l1, l2)} function can be used to traverse multiple lists
element-by-element. Here are some examples:

\begin{pyconcode}
>>> group1 = [1, 10, 100, 1000, 10000]
>>> group2 = [1, 2, 4, 8, 16]
>>> for element1, element2 in zip(group1, group2):
...     print("10s: {}, 2s: {}".format(element1, element2))
10s: 1, 2s: 1
10s: 10, 2s: 2
10s: 100, 2s: 4
10s: 1000, 2s: 8
10s: 10000, 2s: 16

\end{pyconcode}

You could use \pythoninline{for index, (e1, e2, e3) in enumerate(zip(l1, l2, l3))} to iterate through multiple lists and access the current index.

\subsection{Creating New Lists}
One way to create a new list based on a previously exisitng list is to use
a for-loop to check each element of the existing list and append
an element to the new list.
For example, the following function takes a list and returns
a list without any negative numbers and with every non-negative number
replaced by its square root:

\begin{pyconcode}
>>> import math
>>> def sqrt_of_elements(elements):
...     result = []
...     for elem in elements:
...         if elem >= 0:
...             result.append(math.sqrt(elem))
...     return result
...
>>> sqrt_of_elements([-1, -2])
[]
>>> test_list = [-1, 0, 1, 4]
>>> r = sqrt_of_elements(test_list)
>>> r
[0.0, 1.0, 2.0]

\end{pyconcode}

% TODO \section{Sample Program}

\newpage
\section{Exercises}\label{exercises}

% TODO \begin{ex}[easyexercise.py]\end{ex} ?

\begin{ex}[navigate2.py] Modify \texttt{navigate.py} from lab 3 so
  that, instead of performing each action as it is entered, the
  program accepts user input without drawing anything until the ``stop''
  command is given all while ignoring invalid inputs.  After the user
  finishes, the program should run the drawing commands all at once.

  There are many ways to approach this problem.
  One way is to accept the user's input and append the strings to a list.
  For example, if the user enters ``forward, left, 50, forward, right, 20, forward, stop,''
  you might have the following list:
  \begin{python3code}
  ['forward', 'left', '50', 'forward', 'right', '20', 'forward']
  \end{python3code}
\end{ex}

\begin{ex}[statistics107.py]
  Write the functions listed below. They should that take a list as a parameter
  and return the following numbers.

  \begin{description}
  \item[\pythoninline{max(elements)}] maximum element.
  \item[\pythoninline{min(elements)}] minimum element.
  \item[\pythoninline{sum(elements)}] the sum of every element in the list.
  \end{description}

  Although you may find them useful for testing, do not use the
  built-in functions \pythoninline{max, min, sum}.
  Also, write the following functions that take a list and return a list with the
  following elements:

  \begin{description}
  \item[\pythoninline{odds(elements)}] all odd elements of the input list.
  \item[\pythoninline{evens(elements)}] all even elements.
  \item[\pythoninline{every_other(elements)}] finds every other element, starting from the 0$^\text{th}$.
  \item[\pythoninline{every_other_odd(elements)}] finds every other element then returns only the odd elements.
  \item[\pythoninline{every_other_even(elements)}] finds every other element then returns only the even elements.
  \end{description}

  You do not need to accept user input, just write the functions and use the
  function \pythoninline{run_tests} to check them.

  Remember you can call your own functions from any part of your program, even from
  other functions that you write.
  The goal is to reduce repetition.
  What's the minimum amount of code you'll have to write?

  \begin{pyconcode}
>>> import statistics107
>>> statistics107.every_other([1,2,3,4])
[1, 3]
>>> statistics107.every_other_even([1, 100, 45, 23, 10, 2, 4, 13])
[100, 2]
>>> statistics107.evens([1,2,3,4])
[2, 4]
>>> statistics107.sum([1,2,3,4])
10
>>> sum([1,2,3,4])  # write your own!
10

\end{pyconcode}
\end{ex}

% \begin{ex}[sorted.py] Take in numbers as input until ``stop'' is entered. As you
%   take in each number, insert it into a list so that the list is sorted in
%   ascending order. That is, look through the list until you find the place where
%   the new element belongs, then use \pythoninline{.insert()} to place it there.
%   If the number is already in the list, do not add it again. After ``stop'' is
%   entered, print out the list. Do not use any of Python's built-in sorting
%   functions.
%
%   You cannot use \pythoninline!.sort()! for this exercise.
%   % !!! or sorted
%
%   Sample:
%
%   \begin{verbatimcode}
% Input a number >>> 12
% Input a number >>> 5.2
% Input a number >>> 73
% Input a number >>> 45
% Input a number >>> 100
% Input a number >>> -5
% Input a number >>> 2.3
% Input a number >>> stop
% [-5.0, 2.3, 5.2, 12.0, 45.0, 73.0, 100.0]
%   \end{verbatimcode}
%
%   What happens in the background:
%
%   \begin{verbatimcode}
% Input a number >>> 12
% List contains [12.0]
% Input a number >>> 5.2
% List contains [5.2, 12.0]
% Input a number >>> 73
% List contains [5.2, 12.0, 73.0]
% Input a number >>> 45
% List contains [5.2, 12.0, 45.0, 73.0]
% Input a number >>> 100
% List contains [5.2, 12.0, 45.0, 73.0, 100.0]
% Input a number >>> -5
% List contains [-5.0, 5.2, 12.0, 45.0, 73.0, 100.0]
% Input a number >>> 2.3
% List contains [-5.0, 2.3, 5.2, 12.0, 45.0, 73.0, 100.0]
% Input a number >>> stop
% [-5.0, 2.3, 5.2, 12.0, 45.0, 73.0, 100.0]
%   \end{verbatimcode}
%
% \end{ex}

\printindex
\vfill
\section*{Submitting}\label{submitting}
\addcontentsline{toc}{section}{Submitting}

You should submit your code as a tarball that contains all the
exercise files for this lab. The submitted file should be named
\begin{center}
  \texttt{cse107\_firstname\_lastname\_lab\thelabnumber.tar.gz}
\end{center}

\begin{center}
  \textbf{Upload your tarball to Canvas.}
\end{center}

\listofexercises

Exercises start on page~\pageref{exercises}.

\end{document}
